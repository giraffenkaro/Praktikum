\chapter{Visions-Dokument}

\section{Projektbeschreibung}

\subsection{Aufgabenstellung}\label{aufgabe}
Das Projekt umfasst die Entwicklung und Implementierung eines Sudokus.
\newline
\newline
Folgende Aufgaben sollen umgesetzt werden:
\begin{itemize}
  \item \underline{Sudoku-Probleme l�sen:}
  \newline
  Die einfachste L�sungsmethode ist das Brute-Force-Verfahren oder auch
  Exhaustingverfahren, das relativ einfach umzusetzen, aber nicht sehr
  zeiteffizient ist. Finden Sie daher weitere L�sungsverfahren und ermitteln
  Sie, welche(r) Sie sinnvoll einsetzen, ggf. auch in welcher Reihenfolge oder
  fallabh�ngig.
  \item \underline{Sudoku-Probleme erzeugen:}
  \newline
  Generieren Sie ein 9x9-Sudoku. Erm�glichen Sie dabei unterschiedliche
  Schwierigkeitsgrade (in Abg�ngigkeit der Anzahl vorgegebener Ziffern).
\end{itemize}
Das L�sungskonzept ist so umzusetzen, dass m�glichst sowohl das L�sen als auch
das Generieren von 9x9-Spielfeldern zu realisieren ist.

\section{Rahmenbedingung}
Im Rahmen des Moduls ``Praktikum Kommunikationsinformatik'' im Sommersemester
2018 an der htw saar soll ein Projekt nach den Vorgaben in \ref{aufgabe}
umgesetzt werden.
\newline
F�r die Gruppe, mit Jan-Merlin Geuskens als Teamleiter, sind 135 Stunden pro
Teammitglied vorgesehen. Diese dienen der Projektdurchf�hrung. Die
Arbeitsstunden der einzelnen Teammitglieder werden w�chentlich erfasst, ebenso
wie die w�chentliche Arbeitsleistung.

\section{Ziele}
Ziel ist die Entwicklung des Spiels Sudokus mit den dazugeh�rigen L�sungs- und
Generierungsalgorithmen in Bezug auf die vorgegebenen und selbst
definierten Anforderungen f�r das Projekt:
\begin{itemize}
  \item Design der \acs{GUI}
  \item Algorithmen zur L�sung und Generierung
  \item Punktesystem und Highscore
  \item Timer
  \item unterschiedliche Schwierigkeitsgrade
\end{itemize}
