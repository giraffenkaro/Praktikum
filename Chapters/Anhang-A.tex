%********************************************************************
% Appendix
%*******************************************************

\addcontentsline{toc}{part}{Anhang}
\markright{Anhang}

\chapter{Ausz�ge Teil 1}
% 
% eventuell die ganzen Kommandos in den Anhang statt in den Text packen und darauf
% verweisen
% 
% sonstige Ausz�ge zu Part 1 hier rein, auch pings?

\section{Serial Interfaces}

\subsection{Informationen}\label{InfoS}
\begin{lstlisting}
R1#show interfaces serial 0/0/0
Serial0/0/0 is down, line protocol is down
  Hardware is GT96K Serial
  Description: WAN link to R2
  Internet address is 172.17.0.1/16
  MTU 1500 bytes, BW 128 Kbit/sec, DLY 20000 usec,
     reliability 255/255, txload 1/255, rxload 1/255
  Encapsulation HDLC, loopback not set
  Keepalive set (10 sec)
  Last input 00:10:55, output 00:10:56, output hang never
  Last clearing of "show interface" counters 00:18:44
  Input queue: 0/75/0/0 (size/max/drops/flushes); Total output drops: 0
  Queueing strategy: fifo
  Output queue: 0/40 (size/max)
  5 minute input rate 0 bits/sec, 0 packets/sec
  5 minute output rate 0 bits/sec, 0 packets/sec
     93 packets input, 6218 bytes, 0 no buffer
     Received 93 broadcasts (0 IP multicasts)
     0 runts, 0 giants, 0 throttles
     0 input errors, 0 CRC, 0 frame, 0 overrun, 0 ignored, 0 abort
     93 packets output, 6218 bytes, 0 underruns
     0 output errors, 0 collisions, 8 interface resets
     2 unknown protocol drops
     0 output buffer failures, 0 output buffers swapped out
     11 carrier transitions
     DCD=up  DSR=up  DTR=down  RTS=down  CTS=up
\end{lstlisting}

\begin{lstlisting}
R2#show interfaces serial 0/0/0
Serial0/0/0 is up, line protocol is up
  Hardware is GT96K Serial
  Description: WAN link to R1
  Internet address is 172.17.0.2/16
  MTU 1500 bytes, BW 128 Kbit/sec, DLY 20000 usec,
     reliability 255/255, txload 1/255, rxload 1/255
  Encapsulation HDLC, loopback not set
  Keepalive set (10 sec)
  Last input 00:00:06, output 00:00:05, output hang never
  Last clearing of "show interface" counters 00:23:13
  Input queue: 0/75/0/0 (size/max/drops/flushes); Total output drops: 0
  Queueing strategy: fifo
  Output queue: 0/40 (size/max)
  5 minute input rate 0 bits/sec, 0 packets/sec
  5 minute output rate 0 bits/sec, 0 packets/sec
     92 packets input, 5904 bytes, 0 no buffer
     Received 92 broadcasts (0 IP multicasts)
     0 runts, 0 giants, 0 throttles
     0 input errors, 0 CRC, 0 frame, 0 overrun, 0 ignored, 0 abort
     99 packets output, 7259 bytes, 0 underruns
     0 output errors, 0 collisions, 2 interface resets
     0 unknown protocol drops
     0 output buffer failures, 0 output buffers swapped out
     0 carrier transitions
     DCD=up  DSR=up  DTR=up  RTS=up  CTS=up
\end{lstlisting}
\subsection{Verbindung pr�fen}

\begin{lstlisting}
R1#ping 172.17.0.2
Type escape sequence to abort.
Sending 5, 100-byte ICMP Echos to 172.17.0.2, timeout is 2 seconds:
!!!!!
Success rate is 100 percent (5/5), round-trip min/avg/max = 28/28/28 ms
\end{lstlisting}

\begin{lstlisting}
R2#ping 172.17.0.1
Type escape sequence to abort.
Sending 5, 100-byte ICMP Echos to 172.17.0.1, timeout is 2 seconds:
!!!!!
Success rate is 100 percent (5/5), round-trip min/avg/max = 28/28/28 ms
\end{lstlisting}

\section{Fast-Ethernet Interfaces}


\subsection{Verbindung pr�fen}\label{pingF}

\subsubsection{H1 zu R1:}

\begin{lstlisting}
C:\Dokumente und Einstellungen\cisco>ping 172.16.0.1

Ping wird ausgef�hrt f�r 172.16.0.1 mit 32 Bytes Daten:

Antwort von 172.16.0.1: Bytes=32 Zeit=4ms TTL=255
Antwort von 172.16.0.1: Bytes=32 Zeit=1ms TTL=255
Antwort von 172.16.0.1: Bytes=32 Zeit=1ms TTL=255
Antwort von 172.16.0.1: Bytes=32 Zeit=1ms TTL=255

Ping-Statistik f�r 172.16.0.1:
    Pakete: Gesendet = 4, Empfangen = 4, Verloren = 0 (0% Verlust),
Ca. Zeitangaben in Millisek.:
    Minimum = 1ms, Maximum = 4ms, Mittelwert = 1ms
\end{lstlisting}

\subsubsection{H2 zu R2:}

\begin{lstlisting}
C:\Dokumente und Einstellungen\cisco>ping 172.18.0.1

Ping wird ausgef�hrt f�r 172.18.0.1 mit 32 Bytes Daten:

Antwort von 172.18.0.1: Bytes=32 Zeit=2ms TTL=255
Antwort von 172.18.0.1: Bytes=32 Zeit=1ms TTL=255
Antwort von 172.18.0.1: Bytes=32 Zeit=1ms TTL=255
Antwort von 172.18.0.1: Bytes=32 Zeit=1ms TTL=255

Ping-Statistik f�r 172.18.0.1:
    Pakete: Gesendet = 4, Empfangen = 4, Verloren = 0 (0% Verlust),
Ca. Zeitangaben in Millisek.:
    Minimum = 1ms, Maximum = 2ms, Mittelwert = 1ms
\end{lstlisting}
\chapter{Ausz�ge Part 2}

\section{Routingtabellen}

\subsection{Vor RIP-Konfiguration}\label{routeStep6}
\begin{lstlisting}
R1#show ip route
Codes: L - local, C - connected, S - static, R - RIP, M - mobile, B - BGP
       D - EIGRP, EX - EIGRP external, O - OSPF, IA - OSPF inter area
       N1 - OSPF NSSA external type 1, N2 - OSPF NSSA external type 2
       E1 - OSPF external type 1, E2 - OSPF external type 2
       i - IS-IS, su - IS-IS summary, L1 - IS-IS level-1, L2 - IS-IS level-2
       ia - IS-IS inter area, * - candidate default, U - per-user static route
       o - ODR, P - periodic downloaded static route, H - NHRP, l - LISP
       + - replicated route, % - next hop override

Gateway of last resort is not set

      192.168.1.0/24 is variably subnetted, 4 subnets, 3 masks
C        192.168.1.0/25 is directly connected, FastEthernet0/0
L        192.168.1.1/32 is directly connected, FastEthernet0/0
C        192.168.1.192/30 is directly connected, Serial0/0/0
L        192.168.1.193/32 is directly connected, Serial0/0/0
\end{lstlisting}

\begin{lstlisting}
R2#show ip route
Codes: L - local, C - connected, S - static, R - RIP, M - mobile, B - BGP
       D - EIGRP, EX - EIGRP external, O - OSPF, IA - OSPF inter area
       N1 - OSPF NSSA external type 1, N2 - OSPF NSSA external type 2
       E1 - OSPF external type 1, E2 - OSPF external type 2
       i - IS-IS, su - IS-IS summary, L1 - IS-IS level-1, L2 - IS-IS level-2
       ia - IS-IS inter area, * - candidate default, U - per-user static route
       o - ODR, P - periodic downloaded static route, H - NHRP, l - LISP
       + - replicated route, % - next hop override

Gateway of last resort is not set

      192.168.1.0/24 is variably subnetted, 4 subnets, 3 masks
C        192.168.1.128/26 is directly connected, FastEthernet0/0
L        192.168.1.129/32 is directly connected, FastEthernet0/0
C        192.168.1.192/30 is directly connected, Serial0/0/0
L        192.168.1.194/32 is directly connected, Serial0/0/0
\end{lstlisting}

\subsection{Nach RIP-Konfiguration}\label{routeStep9}

\begin{lstlisting}
R1#show ip route
Codes: L - local, C - connected, S - static, R - RIP, M - mobile, B - BGP
       D - EIGRP, EX - EIGRP external, O - OSPF, IA - OSPF inter area
       N1 - OSPF NSSA external type 1, N2 - OSPF NSSA external type 2
       E1 - OSPF external type 1, E2 - OSPF external type 2
       i - IS-IS, su - IS-IS summary, L1 - IS-IS level-1, L2 - IS-IS level-2
       ia - IS-IS inter area, * - candidate default, U - per-user static route
       o - ODR, P - periodic downloaded static route, H - NHRP, l - LISP
       + - replicated route, % - next hop override

Gateway of last resort is not set

      192.168.1.0/24 is variably subnetted, 5 subnets, 4 masks
C        192.168.1.0/25 is directly connected, FastEthernet0/0
L        192.168.1.1/32 is directly connected, FastEthernet0/0
R        192.168.1.128/26 [120/1] via 192.168.1.194, 00:00:22, Serial0/0/0
C        192.168.1.192/30 is directly connected, Serial0/0/0
L        192.168.1.193/32 is directly connected, Serial0/0/0
\end{lstlisting}

\begin{lstlisting}
R2#show ip route
Codes: L - local, C - connected, S - static, R - RIP, M - mobile, B - BGP
       D - EIGRP, EX - EIGRP external, O - OSPF, IA - OSPF inter area
       N1 - OSPF NSSA external type 1, N2 - OSPF NSSA external type 2
       E1 - OSPF external type 1, E2 - OSPF external type 2
       i - IS-IS, su - IS-IS summary, L1 - IS-IS level-1, L2 - IS-IS level-2
       ia - IS-IS inter area, * - candidate default, U - per-user static route
       o - ODR, P - periodic downloaded static route, H - NHRP, l - LISP
       + - replicated route, % - next hop override

Gateway of last resort is not set

      192.168.1.0/24 is variably subnetted, 5 subnets, 4 masks
R        192.168.1.0/25 [120/1] via 192.168.1.193, 00:00:24, Serial0/0/0
C        192.168.1.128/26 is directly connected, FastEthernet0/0
L        192.168.1.129/32 is directly connected, FastEthernet0/0
C        192.168.1.192/30 is directly connected, Serial0/0/0
L        192.168.1.194/32 is directly connected, Serial0/0/0
\end{lstlisting}


\section{RIPv1 vs. RIPv2}

\subsection{Debug Mode - Router 1}

\begin{lstlisting}
R1#debug ip rip
RIP protocol debugging is on
R1#
*May 14 15:08:31.963: RIP: sending v1 update to 255.255.255.255 via FastEthernet
0/0 (192.168.1.1)
*May 14 15:08:31.963: RIP: build update entries - suppressing null update
R1#
*May 14 15:08:53.871: RIP: sending v1 update to 255.255.255.255 via Serial0/0/0
(192.168.1.193)
*May 14 15:08:53.871: RIP: build update entries - suppressing null update
R1#
*May 14 15:08:56.103: RIP: ignored v2 packet from 192.168.1.194 (illegal version
)
R1#
*May 14 15:08:59.251: RIP: sending v1 update to 255.255.255.255 via FastEthernet
0/0 (192.168.1.1)
*May 14 15:08:59.251: RIP: build update entries - suppressing null update
R1#
*May 14 15:09:21.847: RIP: sending v1 update to 255.255.255.255 via Serial0/0/0
(192.168.1.193)
*May 14 15:09:21.847: RIP: build update entries - suppressing null update
*May 14 15:09:22.227: RIP: ignored v2 packet from 192.168.1.194 (illegal version
)
R1#
*May 14 15:09:25.255: RIP: sending v1 update to 255.255.255.255 via FastEthernet
0/0 (192.168.1.1)
*May 14 15:09:25.255: RIP: build update entries - suppressing null update
R1#
*May 14 15:09:48.995: RIP: sending v1 update to 255.255.255.255 via Serial0/0/0
(192.168.1.193)
*May 14 15:09:48.995: RIP: build update entries - suppressing null update
R1#
*May 14 15:09:50.759: RIP: ignored v2 packet from 192.168.1.194 (illegal version
)
R1#
*May 14 15:09:51.831: RIP: ignored v2 packet from 192.168.1.194 (illegal version
)
\end{lstlisting}
\subsection{Debug Mode - Router 2}\label{debugStep12}

\begin{lstlisting}
R2#debug ip rip
RIP protocol debugging is on
R2#
*May 14 15:05:50.547: RIP: sending v2 update to 224.0.0.9 via FastEthernet0/0 (1
92.168.1.129)
*May 14 15:05:50.547: RIP: build update entries
*May 14 15:05:50.547:   192.168.1.0/25 via 0.0.0.0, metric 2, tag 0
*May 14 15:05:50.547:   192.168.1.192/30 via 0.0.0.0, metric 1, tag 0
R2#
*May 14 15:06:03.595: RIP: sending v2 update to 224.0.0.9 via Serial0/0/0 (192.1
68.1.194)
*May 14 15:06:03.595: RIP: build update entries
*May 14 15:06:03.595:   192.168.1.128/26 via 0.0.0.0, metric 1, tag 0
R2#
*May 14 15:06:16.535: RIP: sending v2 update to 224.0.0.9 via FastEthernet0/0 (1
92.168.1.129)
*May 14 15:06:16.535: RIP: build update entries
*May 14 15:06:16.535:   192.168.1.0/25 via 0.0.0.0, metric 2, tag 0
*May 14 15:06:16.535:   192.168.1.192/30 via 0.0.0.0, metric 1, tag 0
R2#undebug all
*May 14 15:06:29.719: RIP: sending v2 update to 224.0.0.9 via Serial0/0/0 (192.1
68.1.194)
*May 14 15:06:29.719: RIP: build update entries
*May 14 15:06:29.719:   192.168.1.128/26 via 0.0.0.0, metric 1, tag 0
\end{lstlisting}