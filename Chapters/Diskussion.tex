\chapter{Diskussion der Messergebnisse und Ausarbeiten der Aufgaben}

Der Laborversuch ``Basic Router Settings'' dient als weiterf�hrende Einf�hrung
zur Konfiguration von Cisco Routern und deren Command Line Interface. Aus dem
bisherigen Studium Erlerntes konnte eingesetzt, vertieft und erweitert werden.
\newline
Dar�ber hinaus veranschaulicht der Versuch praktisch den Aufbau eines
Multirouter-Netzwerks und m�gliche Probleme, die auftreten k�nnen, sowohl bei
der Verkabelung, als auch der Konfiguration.

\section{Verkabelung}

Bei der Verkabelung ist das ``Null serial cable'', welches bisher kein
Gegnstand des Studiums darstellte, zu beachten.
\newline
Das Kabel besteht aus zwei unterschiedlichen Teilen: \ac{DTE} und \ac{DCE}. Der
Hauptunterschied besteht darin, dass das DCE-Ger�t den Takt (clock signal)
vorgibt mit dem die Kommunikation �ber den Bus verl�uft.\cite{cisco4}


\section{Konfiguration}
\ac{RIP} ist u.a. Gegenstand des Moduls ``Protokolle'', fand allerdings noch
keine praktische Anwendung. Daher ist das Szenario aus \ref{szenario2}, bei
denen die Hosts unterschiedliche Versionen von \acs{RIP} nutzen, sehr
anschaulich. Durch den Debug Mode lies sich das Verhalten der Router beobachten.
