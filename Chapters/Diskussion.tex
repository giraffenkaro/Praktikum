\chapter{Diskussion der Messergebnisse}

Der Laborversuch ``Basic Switch Settings'' dient als Einf�hrung in die
Konfiguration von Switches und \acp{VLAN}.

\section{Teil 1}
Teil 1 schafft einen groben �berblick wie VLANs bei Cisco Switches anzulegen und
zu konfigurieren sind. 
Zwar besch�ftigt sich das Modul ``Rechnernetze'', welches Teil des Grundstudiums
ist, mit den Thematiken des Laborversuchs, jedoch nicht allzu konkret mit den
verschiedenen M�glichkeiten der Port Security und wie sich unbefugte Zugriffe
vermeiden lassen.

\section{Teil 2}
Teil 2 besch�ftigt sich vertieft mit VLANs und deren Konfiguration an einem
praktischen Beispiel. 
\newline
Hervorzuheben ist Kapitel \ref{pakete}, das sich mit der Analyse eines Pings und
dessen Weg durchs Netzwerk besch�ftigt. Durch den Mitschnitt und die Analyse der
OSI-Layern 2 und 3 wird deutlich, dass die Konfiguration der VLANs und des VLAN
Trunkings keinen Einfluss auf Layer 3 haben. Lediglich Layer 2, der
Ethernet-Header wird um das VLAN-Tag erweitert. F�r die Hosts bleibt diese
�nderung unbemerkt und das Netzwerk transparent.
