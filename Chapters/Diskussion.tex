\chapter{Diskussion der Messergebnisse und Ausarbeiten der Aufgaben}

Mit dem durchgef�hrten Versuch konnte das Funktionsprinzip einer
Verbindungsauf- und abbauphase innerhalb des Basisanschlusses veranschaulicht
werden.\\Durch die EyeSDN Software war es m�glich einen
D-Kanal-Protokollmitschnitt zu erfassen und hinsichtlich der Rahmen und
Nachrichtentypen zu analysieren.\\Die Versuchsfragen wurden mit Hilfe des
Protokollauschnitts beantwortet. Weiterf�hrend wurden u.a. auch die \acs{ITU-T}
ISDN-Standards verwendet. Die beantworteten Fragen befinden sich zus�tzlich zur
�bersichtlichkeit in Kurzform im Anhang A.
\newline
Anhang B beinhaltet die relevanten Frames auf die sich die Ausarbeitung st�tzt.
Es sind lediglich Mitschnitte der Anrufe, die vom Telefon mit der Rufnummer 100
ausgehen, aufgef�hrt. Zwar wurden aus Interesse auch auch Anrufe in
entgegengesetzter Richtung durchgef�hrt und mitgeschnitten, allerdings sind
diese f�r die Beantwortung der Fragen nicht relevant, da lediglich Telefon 1 mit
dem EyeSDN USB S0 verbunden ist.
\newline
\newline
Die Ergbnisse des Versuchs entsprechen unseren Erwartungen. Hervorzuheben ist
allerdings das Kapitel \ref{setup}. ''Die Technik der Netze'' bildet die 
Bits teilweise in umgekehrter Reihenfolge ab, im Vergleich zu dem
mitgeschnittenen D-Kanal-Protokoll. Ein Abgleich mit dem ITU-T
Standard\cite{recommendation1998digital} f�hrte dann zum dargestellten Ergbenis.
