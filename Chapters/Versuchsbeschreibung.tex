\chapter{Beschreibung des Versuchs}

Der Laborversuch besch�ftigt sich mit einer Einf�hrung in die Konfiguration von
Cisco Routern mit dem Cisco \ac{CLI}.

\section{Versuchsaufbau}\label{Aufbau}

\begin{figure}[ht] 
  \centering
     \includegraphics[width=\linewidth]{Graphics/Aufbau.PNG}
  \caption{Versuchsaufbau}
  \label{fig:aufbau}
\end{figure}

\subsection{Komponenten}

\begin{itemize}
  \item 2 Computer
  \begin{itemize}
    \item HyperTerminal als Emulationsprogramm
  \end{itemize}
  \item 2 Cisco 2811 Router
  \begin{itemize}
    \item Router 1: 
    \begin{itemize}
      \item Verbindung zu PC1 durch Konsolenkabel
      \item Verbindung zum Switch durch Straight-Through-Kabel
	\end{itemize}
	\item Router 2:
	\begin{itemize}
    \item Verbindung zu PC2 durch Crossover-Kabel und Konsolenkabel 
    \end{itemize}
    \item Verbindung untereinander durch Serial-Kabel
  \end{itemize}
  \item 1 Cisco 2960 Switch
  	\begin{itemize}
    	\item Verbindung zu PC1 durch Straight-Through-Kabel
   	\end{itemize}
\end{itemize}

\clearpage

\section{Versuchsdurchf�hrung}

Im ersten Teil des Versuchs wird ein Multi-Router-Netzwerk wie in \ref{Aufbau}
aufgebaut und wie folgt konfiguriert:\\ %TODO Tabelle einf�gen
\newline
\underline{Weiteres Vorgehen:}
\begin{itemize}
  \item Hostnamen der Router konfigurieren
  \item Konsole, ``priviledged EXEC mode'' und vty
  passwords konfigurieren
  \item Ethernet- und Serial-Schnittstellen konfigurieren
  \item \ac{MOTD} konfigurieren
  \item Router konfigurieren, dass keine Adressaufl�sung f�r Hostnamen
  durchgef�hrt wird
  \item Synchrones ``console logging'' konfigurieren
  \item Verbindung der Hosts und Router �berpr�fen
\end{itemize}

Der zweite Teil des Versuchs ist analog durchzuf�hren. %TODO
% aufgabenstellung, Config etc


\section{Versuchsziel}
