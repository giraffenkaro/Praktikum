\chapter{Beschreibung des Versuchs}

Der folgende Versuch besch�ftigt sich mit dem ISDN-Basisanschluss. Anhand einer
Telefonverbindung wird das D-Kanal-Protokoll mitgeschnitten und analysiert.

\section{Versuchsaufbau}\label{Aufbau}

%TODO Bild des versuchsaufbau einf�gen (tats�chliche Ports)

\section{Versuchsdurchf�hrung}

Nach dem Versuchsaufbau wird die Anwendung EyeSDN gestartet und sichergestellt,
dass der Mitschnittdienst aktiviert ist. Der Switchmanager wird ebenfalls
gestartet (Login als Super User).\\Im Anschluss wird durch Starten des ISDN Softswitch
Liberator S1 eine Verbindung zum ISDN Switch hergestellt.
Um die Telefone zu aktivieren werden die ISDN BRI Ports 1 und 2 konfiguriert.
Danach wird das Call Routing eingerichtet. Dies geschieht durch Anlegen eines
Routingprofils f�r die Route von Telefon 1 zu Telefon 2 und ein Profil f�r den
umgekehrten Fall.\\In den Profilen werden die entsprechenden Ports aktiviert und
die MSN-Nummer zu den durchgeroutet werden soll, eingetragen.\\Nach den
beschriebenen Kofigurationen wird ein Anruf get�tigt und ein Teilnehmer
beendet den Anruf durch Auflegen. Der Mitschnitt dieses Telefonats wird im
Anschluss mittels des Analysetools Wireshark genauer betrachtet und analysiert.

%TODO Softswitch beschreiben
%TODO EyeSDN beschreiben

\section{Versuchsziel}

Mit dem beschriebenen Versuchsaufbau soll das Funktionsprinzip
einer Verbindungsauf- und abbauphase innerhalb des Basisanschlusses
veranschaulicht und das D-Kanal-Protokoll einer Telefnverbundung interpretiert
werden. Au�erdem sollen in diesem Zusammenhang die Informationslemente n�her
betrachtet werden.
