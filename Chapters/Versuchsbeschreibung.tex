\chapter{Beschreibung des Versuchs}

Der Laborversuch besch�ftigt sich mit einer Einf�hrung in die Konfiguration von
Cisco Routern mit dem Cisco \ac{CLI}.

\section{Versuchsaufbau}\label{Aufbau}

\begin{figure}[ht] 
  \centering
     \includegraphics[width=\linewidth]{Graphics/Aufbau.PNG}
  \caption{Versuchsaufbau}
  \label{fig:aufbau}
\end{figure}

\subsection{Komponenten}

\begin{itemize}
  \item 2 Computer
  \begin{itemize}
    \item HyperTerminal als Emulationsprogramm
  \end{itemize}
  \item 2 Cisco 2811 Router
  \begin{itemize}
    \item Router 1: 
    \begin{itemize}
      \item Verbindung zu PC1 durch Konsolenkabel
      \item Verbindung zum Switch durch Straight-Through-Kabel
	\end{itemize}
	\item Router 2:
	\begin{itemize}
    \item Verbindung zu PC2 durch Crossover-Kabel und Konsolenkabel 
    \end{itemize}
    \item Verbindung untereinander durch Serial-Kabel
  \end{itemize}
  \item 1 Cisco 2960 Switch
  	\begin{itemize}
    	\item Verbindung zu PC1 durch Straight-Through-Kabel
   	\end{itemize}
\end{itemize}

\clearpage

\section{Versuchsdurchf�hrung}

\subsection{Part 1}
Im ersten Teil des Versuchs wird ein Multi-Router-Netzwerk wie in \ref{Aufbau}
aufgebaut und wie folgt konfiguriert:\\ %TODO Tabelle einf�gen

\begin{table}[!htbp]
\centering
\small
	\begin{tabularx}{1.05\textwidth}{|X|X|X|X|X|}
	\hline
	\textbf{Device} & \textbf{Hostname} & \textbf{Interface} & \textbf{IP-Address} &
	\textbf{Subnet Mask} \\
	\hline
	R1 	& R1  & Serial 0/0/0 & 172.17.0.1 & 255.255.0.0 \\
	\hline
	& & FastEthernet 0/0 & 172.16.0.1 & 255.255.0.0 \\
	\hline
	R2 & R2 &Serial 0/0/0 & 172.17.0.2 & 255.255.0.0 \\
	\hline
	 & & FastEthernet 0/0 & 172.18.0.1	& 255.255.0.0 \\
	\hline
	\end{tabularx}
\caption{Konfiguration der Router}
\label{config}
\end{table}

\underline{Weiteres Vorgehen:}
\begin{itemize}
  \item Hostnamen der Router konfigurieren
  \item Konsole, ``priviledged EXEC mode'' und vty
  passwords konfigurieren
  \item Ethernet- und Serial-Schnittstellen konfigurieren
  \item \ac{MOTD} konfigurieren
  \item Router konfigurieren, dass keine Adressaufl�sung f�r Hostnamen
  durchgef�hrt wird
  \item Synchrones ``console logging'' konfigurieren
  \item Verbindung der Hosts und Router �berpr�fen
\end{itemize}

\subsection{Part 2}\label{subnet}
Der zweite Teil des Versuchs ist analog durchzuf�hren.\\Gegeben ist der
Adressbereich 192.168.1.0/24. Die Anforderungen an das zu gestaltende Netzwerk
sind:
\begin{itemize}
  \item das \acs{LAN} von Router 1 soll �ber IP-Adressen f�r 120 Hosts verf�gen
  \item das \acs{LAN} von Router 2 soll �ber IP-Adressen f�r 60 Host verf�gen
\end{itemize}
Das weitere Vorgehen gestaltet sich �quivalent zum ersten Teil des
Laborversuchs.
Zus�tzlich wird \acs{RIP}-Routing konfiguriert.


\section{Versuchsziel}

Dieser Versuch dient als Einf�hrung zur Netzwerkgestaltung und
-Konfiguration.%TODO erg�nzen
