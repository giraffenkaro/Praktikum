\chapter{Projektr�ckblick}

\section{Erfolgreiche und nicht erfolgreiche Aspekte}

\subsection{Erfolgreiche Aspekte}
% 
% Unser Projekt bietet alle geforderten Funktionen in Form einer, mittels
% \acs{GWT} entwickelten, Webanwendung.\\Im Hinblick auf das Projektmanagament l�sst sich eine
% deutliche Verbesserung zum Vorjahr erkennen. Die Arbeitspakete wurden zeitnah
% nach Bekanntgabe der Aufgabenstellung definiert, Teammeetings fanden
% regelm��iger statt und die Dokumentation wurde gr��tenteils parallel zur
% Entwicklung gehalten.


\subsection{Nicht erfolgreiche Aspekte}


\section{Fazit}

\subsection{Projektergebnis}

Als Projektergebnis k�nnen wir ein funktionierendes und alle Kriterien der
Aufgabenstellung erf�llendes Programm abliefern. Es wurden sowohl die
vorgegebenen Ziele, als auch die selbst definierten Ziele erreicht.

\subsection{Zeitmanagement und Gesamtarbeitszeit}

% Die vorgegebene Gesamtarbeitszeit f�r Phase 1 des Projekts betr�gt 560
% Stunden. Die Gesamtzeit wurde auf Grundlage der Team-Meetings und der
% jeweiligen Arbeitsstunden berechnet. Aus dieser Rechnung geht hervor, dass der
% tats�chliche Zeitaufwand f�r die meisten Teammitglieder mindestens um den
% Faktor 1,4 von der Vorgabe abweicht. Die genauen Arbeitszeiten der einzelnen
% Teammitglieder sind den Liquidit�tsdiagrammen zu entnehmen. Dennoch ist
% anzumerken, dass das Zeitmanagement im Vergleich zu dem Projekt aus ``Verteilte
% Systeme 1'' bedeutend besser zu bewerten ist. Dies ist unter anderem der neuen
% Teamzusammensetzung zu verdanken.\\\\F�r Phase 2 sind 182 Stunden
% Gesamtarbeitszeit angesetzt. Durch die Designentscheidungen aus Phase 1 war es
% uns m�glich den Aufwand f�r Phase 2 recht gering zu halten. Die
% Gesamtarbeitszeit weicht dennoch von der vorgegbenen Zeit ab, da viel Zeit zur
% Optimierung und Bugfixing verwendet wurde.


\section{Abschlie�ende Worte}
% 
% Durch die gemeinsame Arbeit im Team war es uns m�glich, unsere F�higkeiten in
% den Bereichen Projekt- und Zeitmanagement zu verbessern. Zwar konnten nicht alle
% Fehler aus dem letzten Semester vermieden werden, aber das Team war deutlich
% besser in der Lage auf diese zu reagieren. Alle Teammitglieder waren sehr
% motiviert und haben sich gut ins Team eingebracht, wodurch eine angenehme
% Arbeitsatmosph�re entstanden ist.
