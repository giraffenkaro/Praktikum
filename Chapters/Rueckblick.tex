\chapter{Projektr�ckblick}

\section{Erfolgreiche und nicht erfolgreiche Aspekte}

\subsection{Erfolgreiche Aspekte}

Unser Projekt bietet alle geforderten Funktionen und Algorithmen zur L�sung und
Erzeugung von Sudoku-Problemen gem�� der Aufgabenstellung aus \ref{aufgabe}.
Dar�ber hinaus konnten auch die selbst definierten Anforderungen erf�llt werden.

\subsection{Nicht erfolgreiche Aspekte}
Aus Zeitgr�nden war es uns leider nicht mehr m�glich, andere Sudoku-Varianten
als die klassische 9*9 Generierung zu implementieren. Dies liegt unter anderem
daran, dass die Abgabe der dazugeh�rigen Dokumentation um eine Woche vorgezogen
wurde.
\newline
Desweiteren wurde der Aufwand von Freiform-Sudokus untersch�tzt. Die
Implementierung von anderen Sudoku-Formen stand allerdings nicht im Fokus des
Projekts, sondern die Entwicklung und Implementierung des L�sungs- und
Generierungsalgorithmus.

\section{Fazit}

\subsection{Projektergebnis}

Als Projektergebnis k�nnen wir ein funktionierendes und alle Kriterien der
Aufgabenstellung erf�llendes Programm abliefern. Es wurden sowohl die
vorgegebenen Ziele, als auch die selbst definierten Ziele erreicht. Die
erfolgreichen Aspekte �berwiegen deutlich den nicht erfolgreichen Aspekten.

\subsection{Zeitmanagement und Gesamtarbeitszeit}

Die geplante Gesamtarbeitszeit wurde mit 540 Stunden (135 Stunden pro
Teammitglied) angesetzt. Die tats�chliche Arbeitszeit weicht nur minimal von der
geplanten ab.
Die Gesamtzeit wurde auf Grundlage der Team-Meetings und der jeweiligen
Arbeitstunden berechnet. Die genauen Arbeitszeiten der einzelnen Arbeitszeiten
der Teammitglieder ist dem Liquidit�tdiagramm unter \ref{fig:Liquid} zu
entnehmen.


\section{Abschlie�ende Worte}

Durch die gemeinsame Arbeit im Team war es uns m�glich, unsere F�higkeiten in
den Bereichen Projekt- und Zeitmanagement zu verbessern. Alle Teammitglieder
waren sehr motiviert und haben sich gut ins Team eingebracht, wodurch eine
angenehme Arbeitsatmosph�re entstanden ist.
