\chapter{Beschreibung der verschiedenen Messungen und Ergebnisdarstellung}

\section{Konfiguration der Router}
\subsection{Basiskonfiguration}
Zun�chst ist der Hostname zu R1 zu �ndern. Anschlie�end werden die Passw�rter
konfiguriert und ``logging synchronous'' aktiviert.
\begin{lstlisting}
Router>enable
Router#configure terminal
Router(config)# hostname R1
R1(config)#line console 0
R1(config-line)#password cisco
R1(config-line)#login
R1(config-line)#exit
R1(config)#line vty 0 4
R1(config-line)#password cisco
R1(config-line)#login
R1(config)#enable password cisco
R1(config)#enable secret class
R1(config)#exit
R1(config)#line console 0
R1(config-line)#logging synchronous
\end{lstlisting}
F�r die Konsole wird das Passwort ``cisco'', f�r den Priviledged EXEX Mode
``class'', vergeben. Durch ``logging synchronous'' wird sichergestellt, dass
keine Systemnachrichten die Befehlseingabe unterbrechen.
\newline
\newline
Router 2 und Router 3 werden �uivalent konfiguriert.
\subsection{Fast-Ethernet-Schnittstelle}
\subsection{Serial-Schnittstelle}

\section{Konfiguration der Hosts}

\section{�berpr�fung der Konfiguration}

\section{Konfiguration von IPv6-Routing}
\subsection{Statische Routen}
\subsection{Default Routen}
