\chapter{Beschreibung der verschiedenen Messungen und Ergebnisdarstellung}

\section{Part 1}


\subsection{Konfiguration der Hosts}

Zun�chst werden die Host mit statischen IP-Adressen konfiguriert:

\begin{itemize}
  \item Host 1
  	\begin{itemize}
    	\item IP-Adresse: 172.16.0.2
    	\item Subnetz-Maske: 255.255.0.0
    	\item Default Gateway: 172.16.0.1
    \end{itemize}
  \item Host 2
  	\begin{itemize}
    	\item IP-Adresse: 172.18.0.2
    	\item Subnetz-Maske: 255.255.0.0
    	\item Default Gateway: 172.18.0.1
    \end{itemize}
\end{itemize}

\subsection{Konfiguration der Router}

\subsubsection{Basis-Einstellungen}

Bei beiden Routern ist zun�chst der \textbf{Hostname} zu �ndern. Dies geschieht
�ber den Priviledged EXEC Mode. Mit dem Kommando \textit{hostname R1} bzw.
\textit{hostname R2} werden die Hostnamen zu R1 bzw. R2 ge�ndert.
\newline
Danach werden die \textbf{Passw�rter} (console und vty) konfiguriert und
aktiviert.%TODO Ausz�ge
\newline
Es folgt die Konfiguration eines \textbf{\acs{MOTD}}-Banners:
%TODO Auszug
\newline
Um zu gew�hrleisten, dass die Router \textbf{keine Namensaufl�sung via
DNS-Server} betreiben, wird das Kommando \textit{no ip domain lookup} genutzt.
\newline
%TODO logging synchronous

\subsubsection{Running Configuration}

\subsubsection{Konfiguration der Serial Interfaces}

\subsubsection{Konfiguration der Fast-Ethernet Interfaces}




\section{Part 2}


\subsection{Netzwerkplanung}

\subsection{Konfiguration der Router}

\subsection{Konfiguration der Hosts}

\subsection{Konfiguration der Routing Protokolle}


%TODO Verbindungen testen, Routingtabellen, debug RIP, Reflection (Step 12)