\chapter{Beschreibung der verschiedenen Messungen und Ergebnisdarstellung}

\section{Initial Setup eines Switch}

Zun�chst wird der Cisco 2960 Switch, mittels einem RJ-45-to-DB connector console
Kabel, an einen Computer angeschlossen.\\Nach der Verkabelung wird
HyperTerminal, das genutzte Terminal-Emulations-Programm auf dem Computer
gestartet.
\newline
Der genutzte Port, in diesem Fall COM1, wird ausge�hlt und die Parameter f�r die
Terminal Emulation werden nach den Vorgaben des Versuchs konfiguriert. Nach
diesen Einstellungen wird erst der Switch gestartet.
\newline
Um den Switch neu initialisieren zu k�nnen, m�ssen erst die bereits vorhandenen
Konfigurationen gel�scht werden. Dies geschieht �ber den ``privilledged EXEC
mode'' (enable)\footnote{In diesem Dokument werden die genutzten Kommandos
\textit{kursiv} dargestellt.}\\
\newline
Vorgehensweise:
\begin{itemize}
  \item VLAN-Datei l�schen (\textit{delete flash:vlan.dat})
  \item Startup-Config-Datei l�schen (\textit{erase startup-config})
  \item Software neustarten (\textit{reaload})
\end{itemize}

\section{Initial Setup eines Routers}

\section{Informationen des Router Systems}

\section{Packet Tracer}