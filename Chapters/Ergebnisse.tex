\chapter{Beschreibung der verschiedenen Messungen und Ergbnisdarstellung}

\section{Verbindungsauf- und abbau}

%TODO Sequenzdiagramm

\section{Interpretation des Protokolls}

%TODO Fragen und ANtworten in einem Text erl�utern, eventuell passende
% Unterkapitel w�hlen

\section{Nachrichtenelemente}

\begin{tabularx}{\textwidth}{ |X|X|X|X|X|X|X|X|X|X| }
  \hline
  Nachricht  &Aler \newline ting&Call Proc&Con-nect&DISC&In-fo&Re-lease&Rel.
  Comp.&Set-up&Setup Ack \\
  \hline 
  Bearer Capability  &&&&&&&& M &  \\
  \hline
  Cause &&&&&&&&&  \\
  \hline
  Channel Identification  &&&&&&&&&  \\
  \hline
  Process Indicator  &&&&&&&&&  \\
  \hline
  Display  &&&&&&&&&  \\
  \hline
  Date/Time  &&&&&&&&&  \\
  \hline
  Calling Party Number  &&&&&&&&&  \\
  \hline
  Calles Party Number  &&&&&&&&&  \\
  \hline
  Sending Complete  &&&&&&&&&  \\
  \hline
  Facility  &&&&&&&&&  \\
  \hline
  User to User Information  &&&&&&&&&  \\
  \hline
\end{tabularx}


%TODO Tabelle aus Aufgabenteil c erstellen und ausf�llen, kurzer Text dazu (ist
% das immer so blablabla)

\section{SETUP-Nachricht}%bl�der Titel, Alternative finden und eventuell
% andere Nachrichten betrachten?

\begin{tabularx}{\textwidth}{ |X|X|X|X| }
  \hline
  ka & Bit-Nr. & Hex Code & Beschreibung \\
  \hline 
  Schicht-2-Header  & 0111 1110  & 0xFE  & Start-Flag  \\
  \hline
  Schicht-3-Paketkopf  &&&  \\
  \hline
  1. Info Element  &&&  \\
  \hline
  2. Info-Element  &&&  \\
  \hline
  3. Info-Element  &&&  \\
  \hline
  4. Info-Element  &&&  \\
  \hline
  Schicht-2-Trailer  & 0111 1110  &  & Ende-Flag  \\
  \hline
\end{tabularx}
%TODO Tabelle aus Aufgabenteil d & Text
