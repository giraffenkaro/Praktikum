\chapter{Beschreibung der verschiedenen Messungen und Ergebnisdarstellung}

\section{Konfiguration der Router}
Die folgenden Schritte sind bei allen Routern durchzuf�hren.
\subsection{Basiskonfiguration}
Zun�chst ist der Hostname zu R1 zu �ndern. Anschlie�end werden die Passw�rter
konfiguriert und ``logging synchronous'' aktiviert.
\begin{lstlisting}
Router>enable
Router#configure terminal
Router(config)# hostname R1
R1(config)#line console 0
R1(config-line)#password cisco
R1(config-line)#login
R1(config-line)#exit
R1(config)#line vty 0 4
R1(config-line)#password cisco
R1(config-line)#login
R1(config)#enable password cisco
R1(config)#enable secret class
R1(config)#exit
R1(config)#line console 0
R1(config-line)#logging synchronous
\end{lstlisting}
F�r die Konsole wird das Passwort ``cisco'', f�r den Priviledged EXEC Mode
``class'', vergeben. Durch ``logging synchronous'' wird sichergestellt, dass
keine Systemnachrichten die Befehlseingabe unterbrechen.
\newline
Um die Weiterleitung von IPv6-Paketen zu erm�glichen, muss im \acs{CLI} des
Routers das Kommando \textit{ipv6 unicast-routing} ausgef�hrt werden.
\newline
Router 2 und Router 3 werden �uivalent konfiguriert.

\subsection{Fast-Ethernet-Schnittstelle}\label{FE}
Im Configuration Mode wird die Fast-Ethernet-Schnittstelle 0/0 konfiguriert und
die entsprechende IPv6-Adresse, sowie die Link-Local-Adresse, aus \ref{config1}
zugewiesen:
\begin{lstlisting}
R1(config)#interface FastEthernet 0/0
R1(config-if)#description R1 LAN Default Gateway
R1(config-if)#R1(config-if)# ipv6 address 2001:DB8:1:1::1/64
R1(config-if)# ipv6 address FE80::1 link-local
R1(config-if)#no shutdown
R1(config-if)#exit
R1(config)#exit
\end{lstlisting}
\subsection{Serial-Schnittstelle}
Analog zu \ref{FE} wird die Serial-Schnittstelle 0/0 konfiguriert:
\begin{lstlisting}
R1(config)#interface serial 0/0/0
R1(config-if)#description WAN link to R2
R1(config-if)#ipv6 address 2001:DB8:1:A001::1/64
R1(config-if)#clock rate 64000
R1(config-if)#no shutdown
R1(config-if)#exit
R1(config)#exit
\end{lstlisting}

\section{Konfiguration der Hosts}
Den Hosts wird das jeweilige Default Gateway (Link-Local-Adresse) und die
entsprechende statische IPv6-Adresse gem�� \ref{config1} zugewiesen. 
%TODO erkl�ren?

\section{�berpr�fung der Konfiguration}

\subsection{�berpr�fung mittels Ping-Befehlen}
Zun�chst wird mittels Eingabeaufforderung �berpr�ft, ob die Konfiguration der
Hosts korrekt ist.\footnote{siehe Anhang \ref{ipconfig}}


\section{Konfiguration von IPv6-Routing}
\subsection{Statische Routen}
\subsection{Default Routen}
