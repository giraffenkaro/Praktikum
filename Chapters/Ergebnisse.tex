\chapter{Beschreibung der verschiedenen Messungen und Ergbnisdarstellung}

\section{Verbindungsauf- und abbau}

%TODO Sequenzdiagramm

\section{Interpretation des Protokolls}

\subsection{Adressierung der Endger�te}
Im Adressfeld des HDLC-Protokolls befindet sich der \ac{TEI}. Jedem Endger�t ist
ein solcher Identifier zugeordnet um eine logische Verbindung zwischen
Vermittlungsstelle und Endger�t sicherzustellen. Der TEI kennzeichnet die
Schicht-2-Adresse f�r das ISDN-Ger�t.\\%TODO Quelle: Net_IT, s.325
Der TEI-Wert kann entweder durch manuelle Einstellung am Ger�t hinterlegt werden
oder von der Vermittlungsstelle zugeteilt werden (TEI-Vergabe). In diesem
Versuch stellt das Endger�t zuerst eine Identity Request, worauf die Vermittlungsstelle bei Erfolg
mit Identity Assigned antwortet. Die Vergabe erfolgt mittels U-Frame%TODO: mehr
Dem Ger�t wird der TEI 64 zugeteilt.
Dies ist der erste von der Vermittlungsstelle zu vergebende Wert. Die Adressen 1-63
k�nnen eingestellt werden, 64-126 werden von der Vermittlungsstelle verteilt.
TEI 127 stellt die Broadcast-Adresse dar.
\newline
\ac{SAPI} bezeichnet ein weiteres Element des HDLC-Adressfeldes und kennzeichnet
den momentan verwendeten ISDN-Schicht-2-Dienst.%TODO Quelle: siehe oben,s.326
Bei der TEI-Vergabe ist der Wert des SAPI 63. Dem Protokoll zufolge bedeutet
dies ``Layer 2 management procedures''. Es handelt sich hier um die �bertragung
von paketvermittelten Daten.%TODO Qzelle identisch
%TODO Fragen und ANtworten in einem Text erl�utern, eventuell passende
% Unterkapitel w�hlen

\section{Nachrichtenelemente}

\begin{tabularx}{\textwidth}{ |X|X|X|X|X|X|X|X|X|X| }
  \hline
  Nachricht  &Aler \newline ting&Call Proc&Con-nect&DISC&In-fo&Re-lease&Rel.
  Comp.&Set-up&Setup Ack \\
  \hline 
  Bearer Capability  &&&&&&&& M &  \\
  \hline
  Cause &&&&&&&&&  \\
  \hline
  Channel Identification  &&&&&&&&&  \\
  \hline
  Process Indicator  &&&&&&&&&  \\
  \hline
  Display  &&&&&&&&&  \\
  \hline
  Date/Time  &&&&&&&&&  \\
  \hline
  Calling Party Number  &&&&&&&&&  \\
  \hline
  Calles Party Number  &&&&&&&&&  \\
  \hline
  Sending Complete  &&&&&&&&&  \\
  \hline
  Facility  &&&&&&&&&  \\
  \hline
  User to User Information  &&&&&&&&&  \\
  \hline
\end{tabularx}


%TODO Tabelle aus Aufgabenteil c erstellen und ausf�llen, kurzer Text dazu (ist
% das immer so blablabla)

\section{SETUP-Nachricht}%bl�der Titel, Alternative finden und eventuell
% andere Nachrichten betrachten?


\begin{tabularx}{\textwidth}{ |X|X|X|X| }
  \hline
  & Bit-Nr. & Hex Code & Beschreibung \\
  \hline 
  Schicht-2-Header  & 0111 1110  & 0xFE  & Start-Flag  \\
  \hline
  Schicht-3-Paketkopf  &&&  \\
  \hline
  1. Info Element  &&&  \\
  \hline
  2. Info-Element  &&&  \\
  \hline
  3. Info-Element  &&&  \\
  \hline
  4. Info-Element  &&&  \\
  \hline
  Schicht-2-Trailer  & 0111 1110  &  & Ende-Flag  \\
  \hline
\end{tabularx}
%TODO Tabelle aus Aufgabenteil d & Text
